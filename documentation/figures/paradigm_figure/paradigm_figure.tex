\documentclass[12pt]{scrartcl}

\usepackage{lipsum}
\usepackage{amsmath}


% formatting the captions of figures
\usepackage[bf]{caption} % make headings boldfont

% for drawing :-)
\usepackage{tikz} 
\usetikzlibrary{backgrounds} % to have a background layer
\usepackage{xcolor} % defining my own colors
\usetikzlibrary{positioning} % arrows between nodes and relative positions



% for drawing from different lotteries (left/right)
\usepackage{xifthen}

% Define colors
\colorlet{sampleShade}{gray!40}  % shading of sample trials
\colorlet{choiceShade}{blue!40}  % shading of choice trials
\colorlet{choiceCol}{red!70} % color of chosen option



% Some variables for drawing
\newcommand\sideAdj{13}


%--------------------------------------------------%

\begin{document}

% we want random values
\pgfmathsetseed{9999}
\pgfmathdeclarerandomlist{badLottery}{{0}{0}{0}{0}{0}{0}{0}{1}{1}{1}}
\pgfmathdeclarerandomlist{goodLottery}{{1}{1}{1}{1}{1}{1}{1}{0}{0}{0}}




\begin{figure}
\begin{center}
\resizebox{.9\linewidth}{!}{%
\begin{tikzpicture}

% --------------------------------------------------%
%		 Begin with a general case   			   %
% --------------------------------------------------%


% TO DO

% figure part indication
\node [scale=1.5]at (0,20) {\textbf{A)}};

% drawing two initial rectangles as environment
\filldraw [color=black, fill=white](4,15) rectangle (5,16)node[above, scale=1.5]{\textbf{Environment}};
\filldraw [color=black, fill=white](5,15) rectangle (6,16);


% now "left choice" rectangles
\filldraw [color=black, fill=choiceCol](10,17) rectangle (11,18)node[above, scale=1.5]{\textbf{Choose Left}};
\filldraw [color=black, fill=white](11,17) rectangle (12,18);


% now "right choice" rectangles
\filldraw [color=black, fill=white](10,13) rectangle (11,14) node[above, scale=1.5]{\textbf{Choose Right}};
\filldraw [color=black, fill=choiceCol](11,13) rectangle (12,14);



\begin{scope}[on background layer]
% Drawing lines to the choice rectangles
\draw [line width=0.5mm](6,15.5) -- (10.01,17.5);
\draw [line width=0.5mm](6,15.5) -- (10.01,13.5);

% Drawing lines to outcome probabilities 
\draw [line width=0.5mm](12,13.5) -- (16,14.5) node[right, scale=1.5]{$\Pr({1 \mid right})=0.3$};
\draw [line width=0.5mm](12,13.5) -- (16,12.5)node[right, scale=1.5]{$\Pr({0 \mid right})=0.7$};

\draw [line width=0.5mm](12,17.5) -- (16,18.5)node[right, scale=1.5]{$\Pr({1 \mid left})=0.7$};
\draw [line width=0.5mm](12,17.5) -- (16,16.5)node[right, scale=1.5]{$\Pr({0 \mid left})=0.3$};

\end{scope}

% --------------------------------------------------%
%		 Continue with the Bandit Paradigm 		   %
% --------------------------------------------------%



% figure part indication
\node [scale=1.5]at (0,10) {\textbf{B)}};

% figure label
\node [scale=1.5, fill=choiceShade, right] at (8,8) {\textbf{Choice}};

% "direction of time"
\draw [line width=0.5mm,->] (8,8.8) -- (4,0) node [below,scale=1.5] {\textbf{Time}};

% the variables for all rectangles	
\foreach \x/\y/\filColLeft/\filColRight/\labelColLeft/\labelColRight/\action in {
4/8.8/choiceCol/white/black/white/left,
3.5/7.7/choiceCol/white/black/white/left, 
3/6.6/choiceCol/white/black/white/left, 
2.5/5.5/choiceCol/white/black/white/left, 
2/4.4/choiceCol/white/black/white/left, 
1.5/3.3/white/choiceCol/white/black/right, 
1/2.2/choiceCol/white/black/white/right, 
0.5/1.1/white/choiceCol/white/black/right, 
0/0/white/choiceCol/white/black/right}
{

% Draw a random number based on action
\ifthenelse{\equal{\action}{left}}{\pgfmathrandomitem{\outcome}{goodLottery}}{\pgfmathrandomitem{\outcome}{badLottery}};

% background shading
\begin{scope}[on background layer]
\fill [fill=choiceShade](\x-0.15,\y-0.15) rectangle (\x+2.15,\y+1.15);
\end{scope}

% all rectangles in a loop
\filldraw [color=black, fill=\filColLeft](\x,\y) rectangle (\x+1,\y+1) node[midway, color=\labelColLeft] {\textbf{\outcome}};

\filldraw [color=black, fill=\filColRight](\x+1,\y+1) rectangle (\x+2,\y) node[midway, color=\labelColRight] {\textbf{\outcome}};

}



%--------------------------------------------------%
%	      Now the Sampling Paradigm 			      %
%--------------------------------------------------%


% figure part indication
\node [scale=1.5]at (0+\sideAdj,10) {\textbf{C)}};

% figure labels
\node [scale=1.5, fill=sampleShade,right]at (8+\sideAdj,8) {\textbf{Sampling}};

\node [scale=1.5, fill=choiceShade,right]at (5+\sideAdj,1) {\textbf{Choice}};

% direction of time
\draw [line width=0.5mm,->] (8+\sideAdj,8.8) -- (4+\sideAdj,0) node [below,scale=1.5] {\textbf{Time}};

	
% the variables for all rectangles	
\foreach \x/\y/\filColLeft/\filColRight/\labelColLeft/\labelColRight/\action in { 
4+\sideAdj/8.8/choiceCol/white/black/white/left,
3.5+\sideAdj/7.7/white/choiceCol/white/black/right,
3+\sideAdj/6.6/white/choiceCol/white/black/right,
2.5+\sideAdj/5.5/white/choiceCol/white/black/right,
2+\sideAdj/4.4/choiceCol/white/black/white/left,
1.5+\sideAdj/3.3/white/choiceCol/white/black/left,
1+\sideAdj/2.2/choiceCol/white/black/white/left}
{

% Draw a random number based on action
\ifthenelse{\equal{\action}{left}}{\pgfmathrandomitem{\outcome}{goodLottery}}{\pgfmathrandomitem{\outcome}{badLottery}};


% background shading
\begin{scope}[on background layer]
\fill [fill=sampleShade](\x-0.15,\y-0.15) rectangle (\x+2.15,\y+1.15);
\end{scope}

% drawing all the rectangles
\filldraw [color=black, fill=\filColLeft](\x,\y) rectangle (\x+1,\y+1) node[midway, color=\labelColLeft] {\textbf{\outcome}};

\filldraw [color=black, fill=\filColRight](\x+1,\y+1) rectangle (\x+2,\y) node[midway, color=\labelColRight] {\textbf{\outcome}};

}


% drawing the choice rectangles
\filldraw [color=black, fill=choiceCol](0+\sideAdj,0) rectangle (\sideAdj+1,1) node[midway, color=black] {\textbf{1}};

\filldraw [color=black, fill=white](\sideAdj+1,1) rectangle (\sideAdj+2,0) node[midway, color=white] {\textbf{1}};

% background shading of choice rectangle
\begin{scope}[on background layer]
\fill [fill=choiceShade](\sideAdj-0.15,-0.15) rectangle (\sideAdj+2.15,1.15);
\end{scope}

%--------------------------------------------------%
\end{tikzpicture}
} % closing bracket from scaling

\captionsetup{width=.9\linewidth, format=plain}
\caption[Experimental Paradigms]{\lipsum*[2]}\label{fig:paradigms}
\end{center}
\end{figure}
\end{document}